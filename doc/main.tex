% generated by GAPDoc2LaTeX from XML source (Frank Luebeck)
\documentclass[a4paper,11pt]{report}

\usepackage[top=37mm,bottom=37mm,left=27mm,right=27mm]{geometry}
\sloppy
\pagestyle{myheadings}
\usepackage{amssymb}
\usepackage[latin1]{inputenc}
\usepackage{makeidx}
\makeindex
\usepackage{color}
\definecolor{FireBrick}{rgb}{0.5812,0.0074,0.0083}
\definecolor{RoyalBlue}{rgb}{0.0236,0.0894,0.6179}
\definecolor{RoyalGreen}{rgb}{0.0236,0.6179,0.0894}
\definecolor{RoyalRed}{rgb}{0.6179,0.0236,0.0894}
\definecolor{LightBlue}{rgb}{0.8544,0.9511,1.0000}
\definecolor{Black}{rgb}{0.0,0.0,0.0}

\definecolor{linkColor}{rgb}{0.0,0.0,0.554}
\definecolor{citeColor}{rgb}{0.0,0.0,0.554}
\definecolor{fileColor}{rgb}{0.0,0.0,0.554}
\definecolor{urlColor}{rgb}{0.0,0.0,0.554}
\definecolor{promptColor}{rgb}{0.0,0.0,0.589}
\definecolor{brkpromptColor}{rgb}{0.589,0.0,0.0}
\definecolor{gapinputColor}{rgb}{0.589,0.0,0.0}
\definecolor{gapoutputColor}{rgb}{0.0,0.0,0.0}

%%  for a long time these were red and blue by default,
%%  now black, but keep variables to overwrite
\definecolor{FuncColor}{rgb}{0.0,0.0,0.0}
%% strange name because of pdflatex bug:
\definecolor{Chapter }{rgb}{0.0,0.0,0.0}
\definecolor{DarkOlive}{rgb}{0.1047,0.2412,0.0064}


\usepackage{fancyvrb}

\usepackage{mathptmx,helvet}
\usepackage[T1]{fontenc}
\usepackage{textcomp}


\usepackage[
            pdftex=true,
            bookmarks=true,        
            a4paper=true,
            pdftitle={Written with GAPDoc},
            pdfcreator={LaTeX with hyperref package / GAPDoc},
            colorlinks=true,
            backref=page,
            breaklinks=true,
            linkcolor=linkColor,
            citecolor=citeColor,
            filecolor=fileColor,
            urlcolor=urlColor,
            pdfpagemode={UseNone}, 
           ]{hyperref}

\newcommand{\maintitlesize}{\fontsize{50}{55}\selectfont}

% write page numbers to a .pnr log file for online help
\newwrite\pagenrlog
\immediate\openout\pagenrlog =\jobname.pnr
\immediate\write\pagenrlog{PAGENRS := [}
\newcommand{\logpage}[1]{\protect\write\pagenrlog{#1, \thepage,}}
%% were never documented, give conflicts with some additional packages

\newcommand{\GAP}{\textsf{GAP}}

%% nicer description environments, allows long labels
\usepackage{enumitem}
\setdescription{style=nextline}

%% depth of toc
\setcounter{tocdepth}{1}





%% command for ColorPrompt style examples
\newcommand{\gapprompt}[1]{\color{promptColor}{\bfseries #1}}
\newcommand{\gapbrkprompt}[1]{\color{brkpromptColor}{\bfseries #1}}
\newcommand{\gapinput}[1]{\color{gapinputColor}{#1}}


\begin{document}

\logpage{[ 0, 0, 0 ]}
\begin{titlepage}
\mbox{}\vfill

\begin{center}{\maintitlesize \textbf{\textsf{EquiDeg}\mbox{}}}\\
\vfill

\hypersetup{pdftitle=\textsf{EquiDeg}}
\markright{\scriptsize \mbox{}\hfill \textsf{EquiDeg} \hfill\mbox{}}
{\Huge \textbf{A \textsf{GAP} Package for Equivariant Degree Theory\mbox{}}}\\
\vfill

{\Huge Version 0.1\mbox{}}\\[1cm]
{30 November 2018\mbox{}}\\[1cm]
\mbox{}\\[2cm]
{\Large \textbf{ Haopin Wu   \mbox{}}}\\
\hypersetup{pdfauthor= Haopin Wu   }
\end{center}\vfill

\mbox{}\\
{\mbox{}\\
\small \noindent \textbf{ Haopin Wu   }  Email: \href{mailto://psistwu@outlook.com} {\texttt{psistwu@outlook.com}}\\
  Homepage: \href{http://psistwu.sdf.org} {\texttt{http://psistwu.sdf.org}}}\\
\end{titlepage}

\newpage\setcounter{page}{2}
{\small 
\section*{Copyright}
\logpage{[ 0, 0, 1 ]}
 \index{License} {\copyright} 2017-2018 by Haopin Wu

 \textsf{EquiDeg} package is free software; you can redistribute it and/or modify it under the
terms of the \href{http://www.fsf.org/licenses/gpl.html} {GNU General Public License} as published by the Free Software Foundation; either version 2 of the License,
or (at your option) any later version. \mbox{}}\\[1cm]
{\small 
\section*{Acknowledgements}
\logpage{[ 0, 0, 2 ]}
 We appreciate very much all past and future comments, suggestions and
contributions to this package and its documentation provided by \textsf{GAP} users and developers. \mbox{}}\\[1cm]
\newpage

\def\contentsname{Contents\logpage{[ 0, 0, 3 ]}}

\tableofcontents
\newpage

        
\chapter{\textcolor{Chapter }{Introduction}}\label{ch:intro}
\logpage{[ 1, 0, 0 ]}
\hyperdef{L}{X7DFB63A97E67C0A1}{}
{
 The main purpose of the \textsf{EquiDeg} package is to }

           
\chapter{\textcolor{Chapter }{Basic Math Procedures}}\label{ch:basic}
\logpage{[ 2, 0, 0 ]}
\hyperdef{L}{X79D18FC57E06D8C6}{}
{
 We will explain basic math procedures in this package. 
\section{\textcolor{Chapter }{Poset}}\logpage{[ 2, 1, 0 ]}
\hyperdef{L}{X7FF903DB7982F578}{}
{
 

\subsection{\textcolor{Chapter }{IsSortedPoset}}
\logpage{[ 2, 1, 1 ]}\nobreak
\hyperdef{L}{X82AA170986E6A6F7}{}
{\noindent\textcolor{FuncColor}{$\triangleright$\enspace\texttt{IsSortedPoset({\mdseries\slshape list[, func]})\index{IsSortedPoset@\texttt{IsSortedPoset}}
\label{IsSortedPoset}
}\hfill{\scriptsize (operation)}}\\


 checks whether \mbox{\texttt{\mdseries\slshape list}} is a sorted poset. It uses either the partial order specified by \mbox{\texttt{\mdseries\slshape func}}, or the default one \texttt{\texttt{\symbol{92}}{\textless}} when \mbox{\texttt{\mdseries\slshape func}} is not given. }

 

\subsection{\textcolor{Chapter }{TopologicalSort}}
\logpage{[ 2, 1, 2 ]}\nobreak
\hyperdef{L}{X85F7458386BE0833}{}
{\noindent\textcolor{FuncColor}{$\triangleright$\enspace\texttt{TopologicalSort({\mdseries\slshape list[, func]})\index{TopologicalSort@\texttt{TopologicalSort}}
\label{TopologicalSort}
}\hfill{\scriptsize (operation)}}\\


 performs topological sort on \mbox{\texttt{\mdseries\slshape list}}. It uses either the partial order specified by \mbox{\texttt{\mdseries\slshape func}}, or the default one \texttt{\texttt{\symbol{92}}{\textless}} when \mbox{\texttt{\mdseries\slshape func}} is not given. }

 }

 
\section{\textcolor{Chapter }{Lattice}}\logpage{[ 2, 2, 0 ]}
\hyperdef{L}{X7F85E2247E76A7FA}{}
{
 

\subsection{\textcolor{Chapter }{IsLatticeRep}}
\logpage{[ 2, 2, 1 ]}\nobreak
\hyperdef{L}{X7D1CDD6279E62744}{}
{\noindent\textcolor{FuncColor}{$\triangleright$\enspace\texttt{IsLatticeRep\index{IsLatticeRep@\texttt{IsLatticeRep}}
\label{IsLatticeRep}
}\hfill{\scriptsize (Representation)}}\\


 }

 

\subsection{\textcolor{Chapter }{Lattice}}
\logpage{[ 2, 2, 2 ]}\nobreak
\hyperdef{L}{X7F85E2247E76A7FA}{}
{\noindent\textcolor{FuncColor}{$\triangleright$\enspace\texttt{Lattice({\mdseries\slshape IsLatticeRep, list})\index{Lattice@\texttt{Lattice}}
\label{Lattice}
}\hfill{\scriptsize (constructor)}}\\


 }

 

\subsection{\textcolor{Chapter }{MaximalSubElementsLattice}}
\logpage{[ 2, 2, 3 ]}\nobreak
\hyperdef{L}{X79F4F99079F6CB1B}{}
{\noindent\textcolor{FuncColor}{$\triangleright$\enspace\texttt{MaximalSubElementsLattice({\mdseries\slshape lat})\index{MaximalSubElementsLattice@\texttt{MaximalSubElementsLattice}}
\label{MaximalSubElementsLattice}
}\hfill{\scriptsize (attribute)}}\\
\noindent\textcolor{FuncColor}{$\triangleright$\enspace\texttt{MinimalSupElementsLattice({\mdseries\slshape lat})\index{MinimalSupElementsLattice@\texttt{MinimalSupElementsLattice}}
\label{MinimalSupElementsLattice}
}\hfill{\scriptsize (attribute)}}\\


 }

 }

 }

    

\appendix


\chapter{\textcolor{Chapter }{Utilities}}\logpage{[ "A", 0, 0 ]}
\hyperdef{L}{X7DFC90B87DC503A1}{}
{
 Here, we list utilities which are used throughout the package. 
\section{\textcolor{Chapter }{EquiDeg Info Class}}\logpage{[ "A", 1, 0 ]}
\hyperdef{L}{X82990AE587C7AA78}{}
{
 

\subsection{\textcolor{Chapter }{INFO{\textunderscore}LEVEL{\textunderscore}EquiDeg}}
\logpage{[ "A", 1, 1 ]}\nobreak
\hyperdef{L}{X803F9EA37BBB875F}{}
{\noindent\textcolor{FuncColor}{$\triangleright$\enspace\texttt{INFO{\textunderscore}LEVEL{\textunderscore}EquiDeg\index{INFOuScoreLEVELuScoreEquiDeg@\texttt{INF}\-\texttt{O{\textunderscore}}\-\texttt{L}\-\texttt{E}\-\texttt{V}\-\texttt{E}\-\texttt{L{\textunderscore}}\-\texttt{EquiDeg}}
\label{INFOuScoreLEVELuScoreEquiDeg}
}\hfill{\scriptsize (global variable)}}\\


 is the info level regarding the usage of \texttt{Info} (\textbf{Reference: Info}) throughout the package; its default value is 1. Since higher value results in
more detailed feedback, one can increase the value for debugging purpose. }

 

\subsection{\textcolor{Chapter }{InfoEquiDeg}}
\logpage{[ "A", 1, 2 ]}\nobreak
\hyperdef{L}{X795C35687F251D34}{}
{\noindent\textcolor{FuncColor}{$\triangleright$\enspace\texttt{InfoEquiDeg\index{InfoEquiDeg@\texttt{InfoEquiDeg}}
\label{InfoEquiDeg}
}\hfill{\scriptsize (info class)}}\\


 is the default info class throughout the package. Its default info level is 1. }

 }

 
\section{\textcolor{Chapter }{LaTeX Typesettings}}\logpage{[ "A", 2, 0 ]}
\hyperdef{L}{X827F971383CC7746}{}
{
 

\subsection{\textcolor{Chapter }{LaTeXString}}
\logpage{[ "A", 2, 1 ]}\nobreak
\hyperdef{L}{X81CFC38C82E0D642}{}
{\noindent\textcolor{FuncColor}{$\triangleright$\enspace\texttt{LaTeXString({\mdseries\slshape obj})\index{LaTeXString@\texttt{LaTeXString}}
\label{LaTeXString}
}\hfill{\scriptsize (attribute)}}\\


 stores the LaTeX representation of an object. This attribute is supposed to be
assigned manually. }

 

\subsection{\textcolor{Chapter }{LaTeXTypesetting}}
\logpage{[ "A", 2, 2 ]}\nobreak
\hyperdef{L}{X7EFDF2337B312A0A}{}
{\noindent\textcolor{FuncColor}{$\triangleright$\enspace\texttt{LaTeXTypesetting({\mdseries\slshape obj})\index{LaTeXTypesetting@\texttt{LaTeXTypesetting}}
\label{LaTeXTypesetting}
}\hfill{\scriptsize (attribute)}}\\


 returns the LaTeX typesetting of an object. In default, it simply calls \texttt{LaTeXString} (\ref{LaTeXString}). However, it may behave differently for objects of different categories. }

 }

 }

\def\indexname{Index\logpage{[ "Ind", 0, 0 ]}
\hyperdef{L}{X83A0356F839C696F}{}
}

\cleardoublepage
\phantomsection
\addcontentsline{toc}{chapter}{Index}


\printindex

\newpage
\immediate\write\pagenrlog{["End"], \arabic{page}];}
\immediate\closeout\pagenrlog
\end{document}
